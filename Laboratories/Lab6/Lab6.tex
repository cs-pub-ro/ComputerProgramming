%%%%%%%%%%%%%%%%%%%%%%%%%%%%%%%%%%%%%%%%%%%%%%%%%%%%%%%
% MatPlotLib and Random Cheat Sheet
%
% Edited by Michelle Cristina de Sousa Baltazar
%
% http://matplotlib.org/api/pyplot_summary.html
% http://matplotlib.org/users/pyplot_tutorial.html
%
%%%%%%%%%%%%%%%%%%%%%%%%%%%%%%%%%%%%%%%%%%%%%%%%%%%%%%%

\documentclass[a4paper]{article}
\usepackage[landscape]{geometry}
\usepackage{url}
\usepackage{multicol}
\usepackage{amsmath}
\usepackage{amsfonts}
\usepackage{tikz}
\usetikzlibrary{decorations.pathmorphing}
\usepackage{amsmath,amssymb}
\usepackage{hyperref}

\usepackage{colortbl}
\usepackage{xcolor}
\usepackage{mathtools}
\usepackage{amsmath,amssymb}
\usepackage{enumitem}

% Mihnea
\usepackage{textcomp} % \textquotesingle: Racket: '(1 2 3)
\usepackage{couriers}
\usepackage{listings}
\lstset{
	numbers			= left,
	numberstyle		= \tiny,
	numbersep       = 5pt,
	captionpos		= b,
	breaklines		= true,
	basicstyle		= \ttfamily\footnotesize,
	tabsize			= 4,
	escapeinside	= {~}{~},
}
\lstdefinelanguage{Racket}{
  morekeywords=[1]{define, define-syntax, define-macro, lambda, define-stream, stream-lambda},
  morekeywords=[2]{begin, call-with-current-continuation, call/cc,
    call-with-input-file, call-with-output-file, case, cond,
    do, else, for-each, if,
    let*, let, let-syntax, letrec, letrec-syntax,
    let-values, let*-values,
    and, or, not, delay, force,
    quasiquote, quote, unquote, unquote-splicing,
    map, fold, syntax, syntax-rules, eval, environment },
  morekeywords=[3]{import, export},
  alsodigit=!\$\%&*+-./:<=>?@^_~,
  sensitive=true,
  morecomment=[l]{;},
  morecomment=[s]{\#|}{|\#},
  morestring=[b]",
  basicstyle=\footnotesize\ttfamily,
  keywordstyle=\color[rgb]{0,.3,.7},
  commentstyle=\color[rgb]{0.133,0.545,0.133},
  stringstyle={\color[rgb]{0.75,0.49,0.07}},
  upquote=true,
  breaklines=true,
  breakatwhitespace=true,
  literate=*{`}{{`}}{1}
}

\title{Racket}
\usepackage[brazilian]{babel}
\usepackage[utf8]{inputenc}

\advance\topmargin-1.0in
\advance\textheight3in
\advance\textwidth3in
\advance\oddsidemargin-1.5in
\advance\evensidemargin-1.5in
\parindent0pt
\parskip1pt
\newcommand{\hr}{\centerline{\rule{3.5in}{1pt}}}
%\colorbox[HTML]{e4e4e4}{\makebox[\textwidth-2\fboxsep][l]{texto}
\begin{document}

\begin{center}{\huge{\textbf{Matrice. Operaţii cu matrice: adunare, înmulţire. Reprezentarea în memorie.}}}
% {\large Laborator}
\end{center}
\begin{multicols*}{3}

\tikzstyle{mybox} = [draw=black, fill=white, very thick,
    rectangle, rounded corners, inner sep=10pt, inner ysep=10pt]
\tikzstyle{fancytitle} =[fill=black, text=white, font=\bfseries]

% Mihnea
\tikzstyle{mybox_code} = [mybox, draw = orange, fill=sandybrown]
\tikzstyle{fancytitle_code} = [fancytitle, fill = orange]

\definecolor{almond}{rgb}{0.94, 0.87, 0.8}
\definecolor{apricot}{rgb}{0.98, 0.81, 0.69}
\definecolor{atomictangerine}{rgb}{1.0, 0.6, 0.4}
\definecolor{sandybrown}{rgb}{0.96, 0.64, 0.38}
\definecolor{buff}{rgb}{0.94, 0.86, 0.51}

\definecolor{persianred}{rgb}{0.8, 0.2, 0.2}
\definecolor{papayawhip}{rgb}{1.0, 0.94, 0.84}
\tikzstyle{mybox_persianred} = [mybox, draw = persianred, fill=papayawhip]
\tikzstyle{fancytitle_persianred} = [fancytitle, fill = persianred]

\definecolor{whitesmoke}{rgb}{0.96, 0.96, 0.96}
\definecolor{wenge}{rgb}{0.39, 0.33, 0.32}
\tikzstyle{mybox_blue} = [mybox, draw = wenge, fill=whitesmoke]
\tikzstyle{fancytitle_blue} = [fancytitle, fill = wenge]

\definecolor{cerise}{rgb}{0.87, 0.19, 0.39}
\definecolor{mistyrose}{rgb}{1.0, 0.89, 0.88}
\tikzstyle{mybox_cerise} = [mybox, draw = cerise, fill=mistyrose]
\tikzstyle{fancytitle_cerise} = [fancytitle, fill = cerise]

\definecolor{pinegreen}{rgb}{0.0, 0.47, 0.44}
\definecolor{bubbles}{rgb}{0.91, 1.0, 1.0}
\tikzstyle{mybox_pinegreen} = [mybox, draw = pinegreen, fill=bubbles]
\tikzstyle{fancytitle_pinegreen} = [fancytitle, fill = pinegreen]

\definecolor{cream}{rgb}{1.0, 0.99, 0.82}
\definecolor{mikadoyellow}{rgb}{1.0, 0.77, 0.05}
\tikzstyle{mybox_mikadoyellow} = [mybox, draw = mikadoyellow, fill=cream]
\tikzstyle{fancytitle_mikadoyellow} = [fancytitle, fill = mikadoyellow]

\definecolor{cornsilk}{rgb}{1.0, 0.97, 0.86}
\tikzstyle{mybox_orange} = [mybox, draw = orange, fill=cornsilk]
\tikzstyle{fancytitle_orange} = [fancytitle, fill = orange]

\definecolor{aliceblue}{rgb}{0.94, 0.97, 1.0}
\definecolor{seagreen}{rgb}{0.18, 0.55, 0.34}
\tikzstyle{mybox_seagreen} = [mybox, draw = seagreen, fill=aliceblue]
\tikzstyle{fancytitle_seagreen} = [fancytitle, fill = seagreen]

\definecolor{jazzberryjam}{rgb}{0.65, 0.04, 0.37}
\definecolor{almond}{rgb}{0.94, 0.87, 0.8}
\tikzstyle{mybox_jazzberryjam} = [mybox, draw = jazzberryjam, fill=almond]
\tikzstyle{fancytitle_jazzberryjam} = [fancytitle, fill = jazzberryjam]

\definecolor{amaranth}{rgb}{0.9, 0.17, 0.31}
\definecolor{bisque}{rgb}{1.0, 0.89, 0.77}
\tikzstyle{mybox_amaranth} = [mybox, draw = amaranth, fill=bisque]
\tikzstyle{fancytitle_amaranth} = [fancytitle, fill = amaranth]

\definecolor{carminered}{rgb}{1.0, 0.0, 0.22}
\definecolor{blanchedalmond}{rgb}{1.0, 0.92, 0.8}
\tikzstyle{mybox_carminered} = [mybox, draw = amaranth, fill=blanchedalmond]
\tikzstyle{fancytitle_carminered} = [fancytitle, fill = carminered]

\definecolor{midnightgreen}{rgb}{0.0, 0.29, 0.33}
\definecolor{lavendermist}{rgb}{0.9, 0.9, 0.98}
\tikzstyle{mybox_midnightgreen} = [mybox, draw = midnightgreen, fill=lavendermist]
\tikzstyle{fancytitle_midnightgreen} = [fancytitle, fill = midnightgreen]

\definecolor{indigo}{rgb}{0.29, 0.0, 0.51}
\definecolor{isabelline}{rgb}{0.96, 0.94, 0.93}
\tikzstyle{mybox_indigo} = [mybox, draw = indigo, fill=isabelline]
\tikzstyle{fancytitle_indigo} = [fancytitle, fill = indigo]

\definecolor{russet}{rgb}{0.5, 0.27, 0.11}
\definecolor{ivory}{rgb}{1.0, 1.0, 0.94}
\tikzstyle{mybox_russet} = [mybox, draw = russet, fill=ivory]
\tikzstyle{fancytitle_russet} = [fancytitle, fill = russet]

\definecolor{neongreen}{rgb}{0.22, 0.88, 0.08}
\definecolor{splashedwhite}{rgb}{1.0, 0.99, 1.0}
\tikzstyle{mybox_neongreen} = [mybox, draw = neongreen, fill=splashedwhite]
\tikzstyle{fancytitle_neongreen} = [fancytitle, fill = neongreen]

%---------------------------------------------------------------------------------

\begin{tikzpicture}
\node [mybox_persianred] (box){%
    \begin{minipage}{0.3\textwidth}
\textbf
Matricea este o colecţie omogenă şi bidimensională de elemente.

Acestea pot fi accesate prin intermediul a doi indici, numerotaţi, ca şi în cazul vectorilor, începand de la 0. 

Declaraţia unei matrice este de forma:
\begin{lstlisting}[language=Haskell, numbers=none]
<tip_elemente> <nume_matrice>[<dim_1>][<dim_2>];

int mat[5][10];

#define MAX_ELEM 100
float a[MAX_ELEM][MAX_ELEM];
\end{lstlisting}
\textbf
Numărul de elemente ale unei matrice va fi:
\begin{lstlisting}[language=Haskell, numbers=none]
dim1_1 * dim_2
\end{lstlisting}
	\end{minipage}
};

\node[fancytitle_persianred, right=10pt] at (box.north west) {Matrice};
\end{tikzpicture}

\begin{tikzpicture}
\node [mybox_persianred] (box){%
    \begin{minipage}{0.3\textwidth}
\begin{lstlisting}[language=Haskell, numbers=none]
<tip_elemente> <nume_tablou>[<dim_1>][<dim_2>]...[<dim_n>];

int cube[3][3][3];
\end{lstlisting}

	\end{minipage}
};

\node[fancytitle_seagreen, right=10pt] at (box.north west) {Tablouri multidimensionale};
\end{tikzpicture}

\begin{tikzpicture}
\node [mybox_persianred] (box){%
    \begin{minipage}{0.3\textwidth}
\textbf
Standardul C impune ca un tablou să fie memorat într-o zonă continuă de memorie, astfel ca pentru un tabloul de forma:
\begin{lstlisting}[language=Haskell, numbers=none]
T tab[dim1][dim2]...[dimn];
\end{lstlisting}
\textbf
Dimensiunea ocupată în memorie va fi:
\begin{lstlisting}[language=Haskell, numbers=none]
sizeof(T)*dim1*dim2*...*dimn
\end{lstlisting}
\textbf
Vom considera în continuare cazul particular al unui vector vect de lungime n, şi al unui element oarecare al acestuia, de pe pozitia i.

Atunci când întalneşte numele vect, compilatorul va intelege “adresa în memorie de la care începe vectorul vect”. 
\begin{lstlisting}[language=Haskell, numbers=none]
char a[100];
a[0] = 1;
3[a] = 5;
\end{lstlisting}
\textbf
Asignează 5 variabilei de tip char de la adresa:
\begin{lstlisting}[language=Haskell, numbers=none]
3 + a * sizeof(char) = 3 + a
\end{lstlisting}
\textbf
Este echivalentă cu:
\begin{lstlisting}[language=Haskell, numbers=none]
a[3] = 5;
\end{lstlisting}

	\end{minipage}
};

\node[fancytitle_jazzberryjam, right=10pt] at (box.north west) {Reprezentarea in memorie};
\end{tikzpicture}

\begin{tikzpicture}
\node [mybox_amaranth] (box){%
    \begin{minipage}{0.3\textwidth}
\textbf
Declararea unei matrici unitate:
\begin{lstlisting}[language=Haskell, numbers=none]
#define M 20 /* nr maxim de linii si de coloane */
 
int main() {
  float unit[M][M];
  int i,j,n;
 
  printf("nr.linii/coloane: ");
  scanf("%d", &n);
  if (n > M) {
    return;
  }
 
  for (i = 0; i < n; i++) {
    for (j = 0; j < n; j++) {
      if (i != j) {
        unit[i][j]=0;
      } else {
        unit[i][j]=1;
      }
    }
  }
 
  return 0;
}
\end{lstlisting}
\textbf
Citire/scriere de matrice de reali:
\begin{lstlisting}[language=Haskell, numbers=none]
int main() {
  int nl, nc, i, j;
  float a[20][20];
   /* Citire de matrice */
  printf("nr.linii: ");
  scanf("%d", &nl);
  printf("nr.coloane: ");
  scanf("%d", &nc);
  if (nl > 20 || nc > 20) {
    printf("Eroare: dimensiuni > 20 \n");
    return ;
  }
 
  for (i = 0; i < nl; i++) {
    for (j = 0; j < nc; j++) {
      scanf("%f", &a[i][j]);
    }
  }
  /* Afisare matrice */
  for (i = 0; i < nl; i++) {
    for (j = 0; j < nc; j++) {
      printf("%f ",a[i][j]);
    }
    printf ("\n");
  }
  return 0;
}
\end{lstlisting}
    \end{minipage}
};

\node[fancytitle_amaranth, right=10pt] at (box.north west) {Exemple de programre};
\end{tikzpicture}

\begin{tikzpicture}
\node [mybox_orange] (box){%
    \begin{minipage}{0.3\textwidth}
\begin{lstlisting}[language=Haskell, numbers=none]
#include <stdio.h>
 
#define N 100
 
 
void transpose_matrix(int m[N][N], int n)
{
	int i, j, tmp;
 
	for (i = 0; i < n; i++) {
		for (j = i + 1; j < n; j++) {
			tmp = m[i][j];
			m[i][j] = m[j][i];
			m[j][i] = tmp;
		}
	}
}
 
 
void print_matrix(int m[N][N], int n)
{
	int i, j;
 
	for (i = 0; i < n; i++) {
		for (j = 0; j < n; j++) {
			printf("%d ", m[i][j]);
		}
		printf("\n");
	}
	printf("\n");
}
 
 
int main(void)
{
	int i, n = 3;
 
	int V[N][N] = {
		{1, 2, 3},
		{4, 5, 6},
		{7, 8, 9}
	};
 
	print_matrix(V, n);
	transpose_matrix(V, n);
	print_matrix(V, n);
 
	return 0;
}
\end{lstlisting}
    \end{minipage}
};

\node[fancytitle_orange, right=10pt] at (box.north west) {Transpunerea unei matrici patratice};
\end{tikzpicture}


% %---------------------------------------------------------------------------------

% \begin{tikzpicture}
% \node [mybox_persianred] (box){%
%     \begin{minipage}{0.3\textwidth}\centering
% %     	{\centering\bf\color{persianred}
% 		\begin{lstlisting}[language=Racket]
% DA: (cons x L)        NU: (append (list x) L)
% 	                  NU: (append (cons x '()) L)
% DA: (if c vt vf)      NU: (if (equal? c #t) vt vf)
% DA: (null? L)         NU: (= (length L) 0)
% DA: (zero? x)         NU: (equal? x 0)
% DA: test              NU: (if test #t #f)
% DA: (or ceva1 ceva2)  NU: (if ceva1 #t ceva2)
% DA: (and ceva1 ceva2) NU: (if ceva1 ceva2 #f)
%         \end{lstlisting}
%     \end{minipage}
% };

% \node[fancytitle_persianred, right=10pt] at (box.north west) {AȘA  DA / AȘA NU};
% \end{tikzpicture}


% %---------------------------------------------------------------------------------

% % \begin{tikzpicture}
% % \node [mybox_code] (box){%
% %     \begin{minipage}{0.3\textwidth}
% % 		\begin{lstlisting}[language=Racket]

% %         \end{lstlisting}
% %     \end{minipage}
% % };

% % \node[fancytitle_code, right=10pt] at (box.north west) {Cum gandim un program};
% % \end{tikzpicture}

% %---------------------------------------------------------------------------------

% \begin{tikzpicture}
% \node [mybox_orange] (box){%
%     \begin{minipage}{0.3\textwidth}\small
%       \begin{enumerate}\itemsep.1ex
%       	\item După ce variabilă(e) fac recursivitatea? (ce variabilă(e) se schimbă de la un apel la altul?)
%         \item Care sunt condițiile de oprire în funcție de aceste variabile?(cazurile ``de bază'')
%         \item Ce se întâmplă când problema nu este încă elementară? (Obligatoriu aici cel puțin un apel recursiv)
%       \end{enumerate}
%     \end{minipage}
% };
% \node[fancytitle_orange, right=10pt] at (box.north west) {Programare cu funcții recursive};
% \end{tikzpicture}


% %---------------------------------------------------------------------------------

% \begin{tikzpicture}
% \node [mybox_neongreen] (box){%
%     \begin{minipage}{0.3\textwidth}\centering
% \href{http://docs.racket-lang.org/}{http://docs.racket-lang.org/}
%     \end{minipage}
% };

% \node[fancytitle_neongreen, right=10pt] at (box.north west) {Folositi cu incredere!};
% \end{tikzpicture}

%---------------------------------------------------------------------------------

\end{multicols*}
\end{document}
Contact GitHub API Training Shop Blog About
© 2016 GitHub, Inc. Terms Privacy Security Status Help
